\documentclass{article}
\usepackage[spanish]{babel}
\usepackage{mathtools}
\usepackage{amssymb}
\usepackage{amsthm}

\newcommand{\ito}{Itô}

\newcommand{\Expectation}{\mathbb{E}}
\newcommand{\dd}{\mathop{}\!\mathrm{d}}\newcommand{\complexNumbers}{\mathbb{C}}
\DeclarePairedDelimiter{\abs}{\lvert}{\rvert}

\theoremstyle{plain}
\newtheorem{theorem}{Teorema}
\theoremstyle{remark}
\newtheorem{remark}{Observación}
\begin{document}
Una función entera evaluada sobre un movimiento browniano planar es un movimiento browniano.

% Book: Brownian Motion
% Author: Mörters, Peres
% Page Number: 212

Para motivar el resultado supongamos que \(f : \complexNumbers \rightarrow \complexNumbers\) es analítica, es decir diferenciable en el sentido complejo en todas partes, y escribamos \(f = f_1 + i f_2\) para referirnos a la descomposición de \(f\) en partes real e imaginaria.
Entonces, por las ecuaciones de Cauchy--Riemann \(\partial_1 f_1 = \partial f_2\) y \(\partial_2 f_1 = - \partial_1 f_2\), tenemos \(\Delta f_1 = \Delta f_2 = 0\).
Entonces la fórmula de \ito{} (si aplica) afirma que casi seguramente, para todo \(t \geq 0\),
\begin{align}
  f(B(t))
  =
  \int_0^t f'(B(s)) \dd B(s)
\end{align}
donde \(\dd B(s)\) es una abreviación para \(\dd B_1(s) + i \dd B_2(s)\) con \(B(s) = B_1(s) + i B_2(s)\).
El lado derecho define un proceso continuo con incrementos independientes, y es al menos plausible que sean gaussianos.
Más aún, su esperanza es nula y
\begin{align}
  \Expectation \left[ \left( \int_0^t f'(B(s)) \dd B(s) \right)^2 \right]
  =
  \Expectation \int_0^t \abs{f'(B(s))}^2 \dd s
\end{align}
sugiriendo que \(\{f(B(t)) : t \geq 0\}\) es un movimiento Browniano `viajando' con la velocidad alterada
\begin{align}
  t
  \mapsto
  \int_0^t
\end{align}
Para pasar de esta eurística a un poderoso teorema permitimos que la función sea una aplicación analítica \(f : U \rightarrow V\) entre dominios en el plano.
Recordamos que tal aplicación es conforme si es una biyección.

\begin{theorem}
  Sea \(U\) un dominio en el plano complejo, \(x \in U\), y sea \(f : U \rightarrow V\) analítica.
  Sea \(\{B(t) : t \geq 0\}\) un movimiento Browniano que comienza en \(x\) y
  \begin{align}
    \tau_U
    =
    \inf \{ t \geq 0 : B(t) \notin U\}
  \end{align}
  su primer tiempo de salida del dominio \(U\).
  Entonces el proceso \(\{f(B(t)) : 0 \leq t \leq \tau_U\}\) es un movimiento Browniano con velocidad alterada, es decir que existe un movimiento Browniano planar \(\left\{ \widetilde{B}(t) : t \geq 0 \right\}\) tal que, para todo \(t \in \left[ 0, \tau_U \right[\),
  \begin{align}
    & f(B(t)) = \widetilde{B}(\zeta(t))
    &
    & \text{donde }
    \zeta(t) = \int_0^t \abs{f'(B(s))}^2 \dd s
  \end{align}
  Si, adicionalmente, \(f\) es conforme, entonces \(\zeta(\tau_U)\) es el primer tiempo de salida de \(V\) por \(\widehat{B}(t) : t \geq 0\).
\end{theorem}
\begin{remark}
  Notar que, al ser \(f\) diferenciable en el sentido complejo, la derivada \(Df(x)\) no es otra cosa que la multiplicación por el número complejo \(f'(x)\), y \(f\) puede aproximarse localmente alrededor de \(x\) por su tangente \(z \mapsto f(x) + f'(x)(z - x)\).
  La derivada de la alteración temporal es
  \begin{align}
    \partial_t \zeta(t)
    =
    \abs{f'(B(t))}^2
    =
    (\partial_1 f_1(B(t)))^2 + (\partial_2 f_1(B(t)))^2
  \end{align}
   
\end{remark}
\end{document}