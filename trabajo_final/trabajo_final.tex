\documentclass{article}
% Document Packages %
%%%%%%%%%%%%%%%%%%%%%
\usepackage[spanish]{babel}
% Mathematics Packages %
%%%%%%%%%%%%%%%%%%%%%%%%
\usepackage{mathtools}
\usepackage{amssymb}
\usepackage{amsthm}
% Text Abreviation %
%%%%%%%%%%%%%%%%%%%%
\newcommand{\ito}{Itô}
% Mathematics Notation %
%%%%%%%%%%%%%%%%%%%%%%%%
\newcommand{\events}{\mathcal{F}}
\newcommand{\placeholderParameter}{-}
\newcommand{\dd}{\mathop{}\!\mathrm{d}}
\newcommand{\complexNumbers}{\mathbb{C}}
\DeclareMathOperator{\laplacian}{\Delta}
\DeclareMathOperator{\grad}{\nabla}
\DeclareMathOperator{\Expectation}{\mathbb{E}}
% Paired Delimiters %
%%%%%%%%%%%%%%%%%%%%%
\DeclarePairedDelimiter{\abs}{\lvert}{\rvert}
\DeclarePairedDelimiter{\ang}{\langle}{\rangle}
% Theorem environments %
%%%%%%%%%%%%%%%%%%%%%%%%
\theoremstyle{plain}
\newtheorem{theorem}{Teorema}
\theoremstyle{remark}
\newtheorem{remark}{Observación}

\title{Invarianza conforme del movimiento Browniano planar}

\begin{document}
\maketitle

% Archivo: teorica19.pdf %
% Número de Página: 4    %
%%%%%%%%%%%%%%%%%%%%%%%%%%
% Definición de tiempo de parada
\textbf{Tiempo de parada.}
Sea \((\events_t)_{t \geq 0}\) una filtración en un espacio \(\Omega\).
Una función \(\tau : \Omega \rightarrow \left[ 0, \infty \right[\) es un tiempo de parada si \([T \leq t] \in \events_t\) para todo \(t \geq 0\).





\newpage
% Book: Brownian Motion  %
% Author: Mörters, Peres %
% Page Number: 212       %
%%%%%%%%%%%%%%%%%%%%%%%%%%

Para motivar el resultado supongamos que \(f : \complexNumbers \rightarrow \complexNumbers\) es \emph{analítica}, es decir diferenciable en el sentido complejo en todas partes, y escribamos \(f = f_1 + i f_2\) para referirnos a la descomposición de \(f\) en partes real e imaginaria.
Entonces, por las ecuaciones de Cauchy--Riemann \(\partial_1 f_1 = \partial f_2\) y \(\partial_2 f_1 = - \partial_1 f_2\), tenemos \(\Delta f_1 = \Delta f_2 = 0\).
Entonces la fórmula de \ito{} (si aplica) afirma que casi seguramente, para todo \(t \geq 0\),
\begin{align}
  f(B(t))
  =
  \int_0^t f'(B(s)) \dd B(s)
\end{align}
donde \(\dd B(s)\) es una abreviación para \(\dd B_1(s) + i \dd B_2(s)\) con \(B(s) = B_1(s) + i B_2(s)\).
El lado derecho define un proceso continuo con incrementos independientes, y es al menos plausible que sean gaussianos.
Más aún, su esperanza es nula y
\begin{align}
  \Expectation \left[ \left( \int_0^t f'(B(s)) \dd B(s) \right)^2 \right]
  =
  \Expectation \int_0^t \abs{f'(B(s))}^2 \dd s
\end{align}
sugiriendo que \(\{f(B(t)) : t \geq 0\}\) es un movimiento Browniano `viajando' con la velocidad alterada
\begin{align}
  t
  \mapsto
  \int_0^t \abs{f'(B(s))}^2 \dd s
\end{align}
Para pasar de esta eurística a un poderoso teorema permitimos que la función sea una aplicación analítica \(f : U \rightarrow V\) entre dominios en el plano.
Recordamos que tal aplicación es \emph{conforme} si es una biyección.

\begin{theorem}
  \label{theorem:ConformalInvarianceOfPlanarBrownianMotion}
  Sea \(U\) un dominio en el plano complejo, \(x \in U\), y sea \(f : U \rightarrow V\) analítica.
  Sea \(\{B(t) : t \geq 0\}\) un movimiento Browniano que comienza en \(x\) y
  \begin{align}
    \tau_U
    =
    \inf \{ t \geq 0 : B(t) \notin U\}
  \end{align}
  su primer tiempo de salida del dominio \(U\).
  Entonces el proceso \(\{f(B(t)) : 0 \leq t \leq \tau_U\}\) es un movimiento Browniano con velocidad alterada, es decir que existe un movimiento Browniano planar \(\left\{ \widetilde{B}(t) : t \geq 0 \right\}\) tal que, para todo \(t \in \left[ 0, \tau_U \right[\),
  \begin{align}
    & f(B(t)) = \widetilde{B}(\zeta(t))
    &
    & \text{donde }
    \zeta(t) = \int_0^t \abs{f'(B(s))}^2 \dd s
  \end{align}
  Si, adicionalmente, \(f\) es conforme, entonces \(\zeta(\tau_U)\) es el primer tiempo de salida de \(V\) por \(\widehat{B}(t) : t \geq 0\).
\end{theorem}
\begin{remark}
  Notar que, al ser \(f\) diferenciable en el sentido complejo, la derivada \(Df(x)\) no es otra cosa que la multiplicación por el número complejo \(f'(x)\), y \(f\) puede aproximarse localmente alrededor de \(x\) por su tangente \(z \mapsto f(x) + f'(x)(z - x)\).
  La derivada de la alteración temporal es
  \begin{align}
    \partial_t \zeta(t)
    =
    \abs{f'(B(t))}^2
    =
    (\partial_1 f_1(B(t)))^2 + (\partial_2 f_1(B(t)))^2
  \end{align}
\end{remark}
\begin{remark}
  El famoso \emph{teorema de representación conforme de Riemann} enuncia qeu para cualquier par de conjuntos abiertos simplemente conexos \(U, V \subset \complexNumbers\) existe una aplicación conforme \(f : U \rightarrow V\).
  Esto asegura que hay ejemplos del teorema \ref{theorem:ConformalInvarianceOfPlanarBrownianMotion} por montones.
\end{remark}
\begin{proof}
  Notemos que la derivada de \(f\) es nonula excepto por un conjunto contable de puntos, que no tiene un punto límite en \(U\).
  Dado que este conjunto no es tocado por el movimiento Browniano, podemos removerlo de \(U\) y el conjunto resultante aún es abierto.
  Por lo tanto podemos asumir que la derivada de \(f\) no se desvanece sobre \(U\).

  % Esta parte podría irse
  También podemos asumir, sin pérdida de generalidad, que \(f\) es una aplicación entre domninios \emph{acotados}.
  Caso contrario elijamos \(U_n \subseteq K_n \subseteq U\) tales que \(U_n\) es abierto con \(\bigcup_n U_n = U\) y \(K_n\) es compacto, lo cual implica que \(V_n = f(U_n)\) es acotado.
  Entonces el proceso \(\{f(B(t)) : t \leq \tau_{U_n}\}\) es un movimiento browniano con tiempo alterado para todo \(n\), y esto extiende inmediatamente al proceso \(\{f(B(t)) : t \leq \tau_U\}\).

  El principal argumento de la prueba esta basado en la integración estocástica.
  Recordemos que las ecuaciones de Cauchy--Riemann implican que los vectores \(\nabla f_1\), \(\nabla f_2\) son ortogonales entre sí y cada uno de magnitud \(\abs{f'}\).
  Comezamos definiendo para cada \(t \geq 0\), un tiempo de parada
  \begin{align}
    \sigma(t) = \inf \{s \geq 0 : \zeta(s) \geq t\}
  \end{align}
  que representa la inversa de la alteración temporal.
  Sea \(\{\widetilde{B}(t) : t \geq 0\}\) un movimiento Browniano independiente de \(\{B(t) : t \geq 0\}\), y definamos un proceso \(\{W(t) : t \geq 0\}\) mediante
  \begin{align}
    &W(t)
    =
    f(B(\sigma(t) \wedge \tau_U))
      + \widetilde{B}(t) - \widetilde{B}(t \wedge \zeta(\tau_U))
    &&\text{para } t \geq 0
  \end{align}
  A grades trazos, el proceso \(\{f(B(\sigma(t))) : 0 \leq t \leq \zeta(\tau_U)\}\) llega a su punto final en el instante aleatorio \(\zeta(\tau_U)\) y un movimiento Browniano independiente toma su lugar a partir de ese momento.
  Denotamos \(\mathcal{G}(t)\) a la \(\sigma\)--álgebra generada por \(\{W(s) : s \leq t\}\).
  Será suficiente probar que el proceso \(\{W(t) : t \geq 0\}\) es un movimiento Browniano.
  
  % No puedo decir que es obvio, hay que ser caradura ...
  Es obvio que el proceso es continuo casi seguramente y por tanto basta con mostra que sus distribuciones finito--dimensionales coinciden con las del movimiento Browniano.
  Recordando la transformada de Laplace de la distribución normal bivariada, esto es equivalente a mostrar que, para cualesquiera \(0 \leq s \leq t\) y \(\lambda \in \complexNumbers\),
  \begin{align}
    \Expectation\left[ \exp \ang{\lambda, W(t)} \mid \mathcal{G}(s) \right]
    =
    \exp\left( \frac{1}{2} \abs{\lambda}^2 (t - s) + \ang{\lambda, W(s)} \right)
  \end{align}
  donde hemos usado \(\ang{\placeholderParameter , \placeholderParameter}\) para denotar al producto escalar.
  Esto se sigue directamente una vez que mostremos que, para \(x \in U\),
  \begin{align}
    \label{equation:ConformalInvarianceOfPlanarBrownianMotion_MainEquation}
    \Expectation\left[ 
      \exp\ang{\lambda, W(t)} 
      \mid W(s) = f(x)
    \right]
    =
    \exp\left( \frac{1}{2} \abs{\lambda}^2 (t - s) + \ang{\lambda, f(x)} \right)
  \end{align}
  Para simplificar la notación podemos asumirr \(s = 0\).
  Para la prueba, primero evaluamos la experanza con respecto al movimiento Browniano independiente \(\left\{ \widetilde{B}(t) : t \geq 0 \right\}\), lo cual dá
  \begin{align}
    \Expectation &\left[
      \exp \ang{\lambda, W(t) 
      \mid 
      W(0) = f(x)} 
    \right]
    \\&=
    \Expectation_x \exp \left(
      \ang{\lambda, f(B(\sigma(t), \wedge \tau_U))} 
      + \frac{1}{2} \abs{\lambda}^2 (t - \zeta(\sigma(t) \wedge \tau_U))
    \right)
  \end{align}
  Usamos la fórmula multidimensional de \ito para la aplicación acotada
  \begin{align}
    F(x, u)
    =
    \exp\left(
      \ang{\lambda, f(x)}
      + \frac{1}{2} \abs{\lambda}^2 (t - u)
    \right)
  \end{align}
  que está definida en \(U \times \left] - 1, \infty \right[\), ver la Observación 7.16.
  Para preparar esto, observar que \(\partial_{i i} e^g = \partial_{i i} g + (\partial_i g)^2 e^g\) y por lo tanto
  \begin{align}
    \label{equation:LaplacianOfExponential}
    \Delta e^g
    =
    (\Delta g + \abs{\nabla g}^2) e^g
  \end{align}
  Para \(g = \ang{\lambda , f}\) tenemos \(\nabla g = \sum_{i = 1}^2 \lambda_i \nabla f_i\), lo cual impica \(\abs{\nabla g}^2 = \abs{\lambda}^2 \abs{f'}^2\) dado que los vectores \(\nabla f_i\) son ortogonales con norma \(\abs{f'}\).
  Más aún, \(\nabla g = 0\) por la analiticidad de \(f\).
  Aplicar \ref{equation:LaplacianOfExponential} da
  \begin{align}
    \laplacian \exp \ang{\lambda, f(x)}
    =
    \abs{\lambda}^2 \abs{f'(x)}^2 \exp \ang{\lambda, f(x)}
  \end{align}
  Más aún, tenemos
  \begin{align}
    \partial_u \exp\left( \frac{1}{2} \abs{\lambda}^2 (t - u) \right)
    =
    - \frac{1}{2} \abs{\lambda}^2 \exp\left( \frac{1}{2} \abs{\lambda}^2 (t - u)\right)
  \end{align}
  Ahora definimos \(U_n = \{x \in U : \abs{x - y} \geq \frac{1}{n} (\forall y \in \partial U)\}\).
  Entonces \(\abs{f'(x)}\) se encuentra lejos de cero en \(U_n\) y por lo tanto el tiempo de parada \(T = \sigma(t) \wedge \tau_{U_n}\) es acotado.
  La fórmula multidimensional de \ito da, casi seguramente,
  \begin{align}
    F(B(T), \zeta(T))
    &=
    F(B(0), \zeta(0))
    + \int_0^T \grad_x F(B(s), \zeta(s)) \cdot \dd B(s)
    \\
    &+
    \int_0^T \partial_u F(B(s), \zeta(s)) \dd \zeta(s)
    + \frac{1}{2} \int_0^T \laplacian_x F(B(s), \zeta(s)) \dd s
  \end{align}
  Volviendo sobre las ecuaciones para \(\laplacian \exp\), \(\partial_u \exp\), y recordando que \(\dd \zeta(u) = \abs{f'(B(u))}^2 \dd u\) vemos que los dos términos en la segunda línea se cancelan.
  Haciendo uso de la convergencia dominada y el hecho que la integral estocástica tiene esperanza nula, ver el Ejercicio 7.2, obtenemos
  \begin{align}
    \Expectation \left[
      \exp \ang{\lambda, W(t)}
      \mid
      W(0) = f(x)
    \right]
    &=
    \Expectation_x \left[ F(B(\sigma(t) \wedge \tau_U), \zeta(\sigma(t) \wedge \tau_U))\right]
    \\
    &=
    \lim_{n \rightarrow \infty} \Expectation_x \left[F(B(T), \zeta(T))\right]
    \\
    &=
    F(x, 0)
    \\
    &=
    \exp \left( \frac{1}{2} \abs{\lambda}^2 t + \ang{\lambda, f(x)} \right)
  \end{align}
  Esto muestra \eqref{equation:ConformalInvarianceOfPlanarBrownianMotion_MainEquation} y así completa la prueba del enunciado principal.
  Resta observar que, si \(f\) es conforme entonces cuando \(t \uparrow \tau_u\) el punto \(f(B(t))\) converge a un punto en la frontera de \(V\).
  Por lo tanto \(\zeta(\tau_U)\) es el primer tiempo de salida de \(V\) por el proceso \(\{\widetilde{B}(t) : t \geq 0\}\).
\end{proof}

\end{document}